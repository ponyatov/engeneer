\secly{Характеристика профессиональной деятельности}

\begin{enumerate}
  \item Область профессиональной деятельности ИЦП включает в себя: 
  \begin{itemize}[nosep]
    
    \item \termdef{мехатронику}{мехатроника}\ --- область науки и техники,
    основанная на объединении узлов точной механики, датчиков, исполнительных
    механизмов, источников энергии, услителей и вычислительных устройств, между
    которыми осуществляется динамический обмен энергией и информацией, для
    выполнения целевой задача с применением средств автоматического управления,
    вычислительной техники и программного обеспечения, обладающего элементами
    искуственного интеллекта.
     
    \item \termdef{робототехнику}{робототехника}\ --- область науки и техники,
    ориентированная на создание робототв и робототеъхнических систем,
    построенных на базе мехатронных компонентов, для выполнения рабочих
    операций, прежде всего для замены или дополнения человека на тяжелых,
    утомительных и опасных работах.
    
    \item совокупность средств, способов и методов науки и техники, направленных
    на автоматизацию действующих и создание новых автоматизированных и
    автоматических технологий и производств.
    
    \item разработку, реализацию и контроль мехатронных компонентов, средств 
    производства, программного обеспечения и технологии на всем жизненном цикле
    разработки, изготовления, управления качеством, транспортировки,
    эксплуатации и утилизации на основе стандартов и нормативных документов.
    
    \item информационно-технологическая поддержка проектирования, испытаний,
    логистики, поддержки в течение жизненного цикла и технологических процессов
    изготовления узлов и агрегатов мехатронных систем и комплексов.
    
    \item исследования, создание и применение программно-аппаратных средств
    обеспечения систем автоматизации, управления и контроля технологических
    процессов изготовления, логистики, эксплуатации и ремонта элементов
    мехатронных систем и комплексов.
    
    \item обеспечение экономичного и высокоэффективного функционирования средств
    автоматизации, управления, контроля, испытаний и управления жизненным циклом
    при соблюдении правил эксплуатации и безопасности.
     
  \end{itemize}
\end{enumerate}
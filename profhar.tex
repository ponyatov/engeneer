\secly{Характеристика профессиональной деятельности}

\begin{enumerate}
  \item Область профессиональной деятельности ИЦП включает в себя: 
  \begin{itemize}[nosep]
    
    \item \termdef{мехатронику}{мехатроника}\ --- область науки и техники,
    основанная на объединении узлов точной механики, датчиков, исполнительных
    механизмов, источников энергии, услителей и вычислительных устройств, между
    которыми осуществляется динамический обмен энергией и информацией, для
    выполнения целевой задача с применением средств автоматического управления,
    вычислительной техники и программного обеспечения, обладающего элементами
    искуственного интеллекта.
     
    \item \termdef{робототехнику}{робототехника}\ --- область науки и техники,
    ориентированная на создание робототв и робототеъхнических систем,
    построенных на базе мехатронных компонентов, для выполнения рабочих
    операций, прежде всего для замены или дополнения человека на тяжелых,
    утомительных и опасных работах.
    
    \item совокупность средств, способов и методов науки и техники, направленных
    на автоматизацию действующих и создание новых автоматизированных и
    автоматических технологий и производств.
    
    \item разработку, реализацию и контроль мехатронных компонентов, средств 
    производства, программного обеспечения и технологии на всем жизненном цикле
    разработки, изготовления, управления качеством, транспортировки,
    эксплуатации и утилизации на основе стандартов и нормативных документов.
    
    \item информационно-технологическая поддержка проектирования, испытаний,
    логистики, поддержки в течение жизненного цикла и технологических процессов
    изготовления узлов и агрегатов мехатронных систем и комплексов.
    
    \item исследования, создание и применение программно-аппаратных средств
    обеспечения систем автоматизации, управления и контроля технологических
    процессов изготовления, логистики, эксплуатации и ремонта элементов
    мехатронных систем и комплексов.
    
    \item обеспечение экономичного и высокоэффективного функционирования средств
    автоматизации, управления, контроля, испытаний и управления жизненным циклом
    при соблюдении правил эксплуатации и безопасности.
    
    \item обеспечение подбора, обучения и регулярного повышения квалификации и
    профессиональной переподготовки собственного персонала и сотрудников
    клиентов, эксплуатирующих и обслуживающих сложное высокотехнологическое
    оборудование, и выполняющих проектирование технологических процессов с
    применением мехатронных компонентов, комплексов и систем.
    
    \item подготовка конструкторской, учебной и справочной документации с
    применением систем электронного документооборота, верстки и Web-технологий.
    
    \item разработка, сопровождение и применение средств автоматизации
    проектирования, электронного документооборота, технологической подготовки,
    управления производством, и электронного дистанционного образования
     
  \end{itemize}
  
  \item Объектами профессиональной деятельности инженера по специальности
  \textbf{100500 Инженер цифрового производства}\ являются базирующиеся на
  мехатронных модулях и роботах:
  \begin{itemize}[nosep]
    \item автоматические и автоматизированные технологические системы
    \item средства управления и контроля
    \item математическое, программное и аппаратно-информационные обеспечение
    проектирования, производства, испытания и мониторинга мехатронных систем и
    компонентов
    \item способы и методы проектирования, производства, отладки и эксплуатации
    \item применение мехатронных компонентов в широком диапазоне технологических
    процессов и масштаба производства, включая индивидуальное применение
    частными лицами
    \item обеспечение поддержки мехатронных систем, включая обучение персонала,
    подготовкку сопроводительной документации и систем электронного обучения
  \end{itemize}
  
  
  \item Инженер по специальности \textbf{100500 Инженер цифрового производства}\
  готовится  кследующим видам профессиональной деятельности:
  \begin{itemize}[nosep]
    \item научно-исследовательская
    \item проектно-конструкторская
    \item эксплуатационная
    \item разработка и сопровождение информационно-аппаратных систем
    \item учебная и методическая
  \end{itemize}
  
  Конкретные виды профессиональной деятельности определяются учебныым заведением
  совместо с заинтересованными работодателями в индивидуальном порядке.
  
  \item Инженер по специальности \textbf{100500 Инженер цифрового производства}\
  дожен быть подготовлен к практическому решению профессиональных задач
  \emph{непосредственно после окончания учебного курса, без дополнительного
  обучения со стороны работодателя} по следующим направлениям:
  \begin{itemize}[nosep]
    \item теоретические и экспериментальные исследования, проводимые в целях
    изыскания принципов и путей создания новых объекто профессиональной
    деятельности (далее\ --- изделий), обоснования их технических характеристик,
    определения условий применения, эксплуатации и ремонта
    \item поиск и анализ научно-технической информации по исследуемым
    направлениям, опреление напрявления и методов исследований
    \item разработка \emph{и изготовление}\ экспериментальных образцов изделий
    при выполнении НИР для проверки и обоснования технических решений,
    технологии производства, параметров и эксплуатационных характеристик изделия
    \item выбор средств проектирования, изготовления, испытаний и поддержки
    жизненного цикла изделия, средств контроля изделия и его составных частей в
    процессе изготовления, испытаний и эксплуатации
    \item прогнозирование надежности вариантов изделия по резальтатам
    расчетно-теоретических и экспериментальных работ
    \item корректировка технологии производства и испытаний изделий с учетом
    данных по эксплуатации изделий и оценке себестоимости
    \item разработка и корректировка проектно-конструкторской, программной,
    технологической документации и учебно-методического материала
    \item выбор средств программного обеспечения, инструментария САПР и
    разработчика ПО, библиотек и ПО поддержки с учетом себестоимости разработки,
    сопровождения, внесения модификаций в течение всего срока службы изделия
    \item подготовка и адаптация учебно-тренировочных средств и информационного
    наполнения систем электронного обучения и информационного сопровождения
    изделия
    \item сопровождение аппаратно-программных систем поддержки жизненного чикла
    изделия на всех этапах
     
  \end{itemize}
  
  \item 
	  
    
\end{enumerate}